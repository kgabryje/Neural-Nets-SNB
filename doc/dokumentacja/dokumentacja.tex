\documentclass[a4paper,titlepage,11pt,twosides,floatssmall]{mwrep}
\usepackage[left=2.5cm,right=2.5cm,top=2.5cm,bottom=2.5cm]{geometry}
\usepackage[OT1]{fontenc}
\usepackage{polski}
\usepackage{amsmath}
\usepackage{xr}
\usepackage{amsfonts}
\usepackage{amssymb}
\usepackage{graphicx}
\usepackage{url}
\usepackage{makecell}
\usepackage[section]{placeins}
\usepackage[utf8]{inputenc}
\usepackage{tikz}
\usetikzlibrary{arrows,calc,decorations.markings,math,arrows.meta}
\usepackage{rotating}
\usepackage[percent]{overpic}
\usepackage{xcolor}
\usepackage{graphicx} % Required for including images
\usepackage[font=small,labelfont=bf]{caption} % Required for specifying captions to tables and figures
\usepackage{pgfplots}
\usetikzlibrary{pgfplots.groupplots}
\usepackage{listings}
\usepackage{matlab-prettifier}
\usepackage{siunitx}
\definecolor{szary}{rgb}{0.95,0.95,0.95}
\sisetup{detect-weight,exponent-product=\cdot,output-decimal-marker={,},per-mode=symbol,binary-units=true,range-phrase={-},range-units=single}

%konfiguracje pakietu listings
\lstset{
	backgroundcolor=\color{szary},
	frame=single,
	breaklines=true,
}
\lstdefinestyle{customlatex}{
	basicstyle=\footnotesize\ttfamily,
	%basicstyle=\small\ttfamily,
}
\lstdefinestyle{customc}{
	breaklines=true,
	frame=tb,
	language=C,
	xleftmargin=0pt,
	showstringspaces=false,
	basicstyle=\small\ttfamily,
	keywordstyle=\bfseries\color{green!40!black},
	commentstyle=\itshape\color{purple!40!black},
	identifierstyle=\color{blue},
	stringstyle=\color{orange},
}
\lstdefinestyle{custommatlab}{
	captionpos=t,
	breaklines=true,
	frame=tb,
	xleftmargin=0pt,
	language=matlab,
	showstringspaces=false,
	%basicstyle=\footnotesize\ttfamily,
	basicstyle=\scriptsize\ttfamily,
	keywordstyle=\bfseries\color{green!40!black},
	commentstyle=\itshape\color{purple!40!black},
	identifierstyle=\color{blue},
	stringstyle=\color{orange},
}

%wymiar tekstu (bez ?ywej paginy)
\textwidth 160mm \textheight 247mm

%ustawienia pakietu pgfplots
\pgfplotsset{
tick label style={font=\scriptsize},
label style={font=\small},
legend style={font=\small},
title style={font=\small}
}

\def\figurename{Rys.}
\def\tablename{Tab.}

%konfiguracja liczby p?ywaj?cych element?w
\setcounter{topnumber}{0}%2
\setcounter{bottomnumber}{3}%1
\setcounter{totalnumber}{5}%3
\renewcommand{\textfraction}{0.01}%0.2
\renewcommand{\topfraction}{0.95}%0.7
\renewcommand{\bottomfraction}{0.95}%0.3
\renewcommand{\floatpagefraction}{0.35}%0.5

\begin{document}
\raggedbottom
\frenchspacing
\pagestyle{uheadings}

%strona tytu?owa
\title{\bf Sprawozdanie \vskip 0.1cm}
\author{Kamil Gabryjelski, Antoni Różański}
\date{2017}

\makeatletter
\renewcommand{\maketitle}{\begin{titlepage}
\begin{center}{\LARGE {\bf
Wydział Elektroniki i Technik Informacyjnych}}\\
\vspace{0.4cm}
{\LARGE {\bf Politechnika Warszawska}}\\
\vspace{0.3cm}
\end{center}
\vspace{5cm}
\begin{center}
{\bf \LARGE Sieci neuronowe w zastosowaniach biomedycznych \vskip 0.1cm}
\end{center}
\vspace{1cm}
\begin{center}
\vspace{1cm}
\begin{center}
{\bf \LARGE Rozpoznawanie cyfr pisanych ręcznie ze zbioru MNIST \vskip 0.1cm}
\end{center}
\vspace{1cm}
{\bf \LARGE \@title}
\end{center}
\vspace{4cm}
\begin{center}
{\bf \Large \@author \par}
\end{center}
\vspace{1cm}
\begin{center}
{\bf \Large Prowadzący: mgr inż. Piotr Płoński}
\end{center}
\vspace*{\stretch{6}}
\begin{center}
\bf{\large{Warszawa, \@date\vskip 0.1cm}}
\end{center}
\end{titlepage}
}
\makeatother

\maketitle
\chapter{Dane}
Mieliśmy do dyspozycji $70000$ próbek cyfr pisanych ręcznie, każda składająca się z 784 pikseli. Dane podzielone zostały na:
\begin{itemize}
\item zbiór uczący - $55000$ próbek
\item zbiór testowy - $10000$ próbek
\item zbiór walidacyjny - $5000$ próbek
\end{itemize}
{\let\clearpage\relax \chapter{Implementacja sieci}}
\section{Funkcje aktywacji}
Przetestowanymi przez nas funkcjami aktywacji neuronów są funkcja sigmoidalna(wykres \ref{fig:sig}) oraz ReLU, która przedstawiona jest na wykresie \ref{fig:relu}.

\begin{figure}
\centering
\begin{tikzpicture}
\begin{axis}[
width=0.8\textwidth,
xlabel={x},
ylabel={sig(x)},
/pgf/number format/.cd,
use comma,
1000 sep={}
]
\addplot[blue,semithick] file {wykresy/sig.txt};
\end{axis}
\end{tikzpicture}
\caption{Sigmoidalna funkcja aktywacji}
\label{fig:sig}
\end{figure}

\begin{figure}
\centering
\begin{tikzpicture}
\begin{axis}[
width=0.8\textwidth,
xlabel={x},
ylabel={$ReLU(x) = max(0,x)$},
/pgf/number format/.cd,
use comma,
1000 sep={}
]
\addplot[blue,semithick] file {wykresy/relu.txt};
\end{axis}
\end{tikzpicture}
\caption{Funkcja aktywacji ReLU}
\label{fig:relu}
\end{figure}

\section{Struktura sieci}
Warstwa wejściowa składa się z 784 neuronów, po jednym na każdy piksel obrazka z cyfrą. W trakcie eksperymentów zbadane zostały struktury sieci z dwiema i trzema warstwami ukrytymi. Warstwę wyjściową tworzy 10 neuronów, gdyż oczekujemy, że sieć zwróci jedną z dziesięciu cyfr. Na warstwie wyjściowej używana jest funkcja $softmax$, dana wzorem \ref{eq:softmax}, dzięki której wyniki możemy interpretować jako prawdopodobieństwa.
\begin{equation} \label{eq:softmax}
\sigma(x)_j = \frac{e^{x_j}}{\sum\limits_{i=1}^n e^{x_i}}
\end{equation}

\section{Błąd sieci}
W celu obliczenia błędu warstwy wyjściowej stosowana jest metoda cross entropy, dana wzorem \ref{eq:crossentropy}.
\begin{equation} \label{eq:crossentropy}
L(w) = -\frac{1}{N} \sum\limits_{n=1}^N [y_nlog\hat{y}_n + (1-y_n)log(1-\hat{y}_n)]
\end{equation}

\section{Proces uczenia}
Testowanymi algorytmami uczącymi są stochastyczny spadek gradientu oraz jego modyfikacja wykorzystująca pęd. Ponadto, zastosowaliśmy technikę $dropout$ polegającą na usuwaniu (zerowaniu) losowych połączeń między neuronami sąsiadujących warstw. Ma to na celu zapobiegnięcie zjawisku dopasowywania się sieci do danych uczących. Stosowane przez nas prawdopodobieństwo zachowania połączenia wynosi $0,95$. 

Zastosowaliśmy technikę wykładniczego spadku wartości kroku, dzięki czemu możemy użyć dużej wartości początkowej kroku. Sprawia to, że sieć uczy się szybko na początku eksperymentu, a zwalnia gdy jest w pobliżu optymalnego rozwiązania.
\chapter{Eksperymenty}
Badane będą różne  struktury sieci, funkcje aktywacji, algorytmy uczenia, szybkości uczenia oraz wielkości serii (dalej nazywane batch size).

Ze względu na niedeterministyczny charakter sieci neuronowych, każdy zestaw parametrów był testowany kilkukrotnie i wybierany był najlepszy z uzyskanych wyników.

\section{Struktura sieci}
Wyniki przedstawione w tabeli \ref{table:struktura} pozwalają stwierdzić, że różnice w jakości rozpoznawania cyfry przez sieć w zależności od jej struktury są niewielkie. Sieć prawidłowo rozpoznaje cyfrę w około $98\%$ przypadków. Analizując wykresy błędów można jednak zauważyć, że im więcej neuronów w sieci, tym mniejsza wariancja funkcji błędu. Tę zależność wyraźnie widać, porównująć przykładowo wykresy \ref{fig:w_2_n_100_100} i \ref{fig:w_2_n_400_400}. Biorąc tę obserwację pod uwagę, w kolejnych eksperymentach używana będzie struktura sieci z 2 warstwami ukrytymi po 400 neuronów w każdej.
\begin{table}[h]
\centering
\begin{tabular}{|c|c|c|}
\hline
Ilość warstw ukrytych & Ilość neuronów & Wyniki \\ \hline
2 & 100, 50 & \makecell{Zbiór uczący: 97,7\% \\ Zbiór testowy: 97,5\%} \\ \hline
2 & 100, 100 & \makecell{Zbiór uczący: 99,2\% \\ Zbiór testowy: 97,8\%} \\ \hline
2 & 100, 150 & \makecell{Zbiór uczący: 97,7\% \\ Zbiór testowy: 97,9\%} \\ \hline
2 & 200, 100 & \makecell{Zbiór uczący: 98,4\% \\ Zbiór testowy: 97,9\%} \\ \hline
2 & 200, 200 & \makecell{Zbiór uczący: 98,4\% \\ Zbiór testowy: 98\%} \\ \hline
2 & 200, 300 & \makecell{Zbiór uczący: 97,7\% \\ Zbiór testowy: 97,9\%} \\ \hline
2 & 400, 200 & \makecell{Zbiór uczący: 98,4\% \\ Zbiór testowy: 98\%} \\ \hline
2 & 400, 400 & \makecell{Zbiór uczący: 98,4\% \\ Zbiór testowy: 98\%} \\ \hline
2 & 400, 600 & \makecell{Zbiór uczący: 97,7\% \\ Zbiór testowy: 97,9\%} \\ \hline

3 & 100, 50, 10 & \makecell{Zbiór uczący: 96,9\% \\ Zbiór testowy: 97,7\%} \\ \hline
3 & 100, 100, 50 & \makecell{Zbiór uczący: 97,7\% \\ Zbiór testowy: 98\%} \\ \hline
3 & 100, 100, 100 & \makecell{Zbiór uczący: 97,7\% \\ Zbiór testowy: 97,7\%} \\ \hline
3 & 200, 100, 100 & \makecell{Zbiór uczący: 97,7\% \\ Zbiór testowy: 98\%} \\ \hline
3 & 200, 200, 100 & \makecell{Zbiór uczący: 98,4\% \\ Zbiór testowy: 97,9\%} \\ \hline
3 & 200, 200, 200 & \makecell{Zbiór uczący: 99,2\% \\ Zbiór testowy: 98,1\%} \\ \hline
3 & 400, 200, 200 & \makecell{Zbiór uczący: 97,7\% \\ Zbiór testowy: 98,1\%} \\ \hline
\end{tabular}
\caption{Dane po 5000 kroków}
\label{table:struktura}
\end{table}

\begin{figure}
\centering
\begin{tikzpicture}
\begin{axis}[
width=0.8\textwidth,
xlabel={kroki},
ylabel={Błąd},
/pgf/number format/.cd,
use comma,
1000 sep={}
]
\addplot[blue,semithick] file {wykresy/w_2_n_100_50.txt};
\end{axis}
\end{tikzpicture}
\caption{2 warstwy ukryte, (100, 50) neuronów}
\label{fig:w_2_n_100_50}
\end{figure}

\begin{figure}
\centering
\begin{tikzpicture}
\begin{axis}[
width=0.8\textwidth,
xlabel={kroki},
ylabel={Błąd},
/pgf/number format/.cd,
use comma,
1000 sep={}
]
\addplot[blue,semithick] file {wykresy/w_2_n_100_100.txt};
\end{axis}
\end{tikzpicture}
\caption{2 warstwy ukryte, (100, 100) neuronów}
\label{fig:w_2_n_100_100}
\end{figure}

\begin{figure}
\centering
\begin{tikzpicture}
\begin{axis}[
width=0.8\textwidth,
xlabel={kroki},
ylabel={Błąd},
/pgf/number format/.cd,
use comma,
1000 sep={}
]
\addplot[blue,semithick] file {wykresy/w_2_n_100_150.txt};
\end{axis}
\end{tikzpicture}
\caption{2 warstwy ukryte, (100, 150) neuronów}
\label{fig:w_2_n_100_150}
\end{figure}

\begin{figure}
\centering
\begin{tikzpicture}
\begin{axis}[
width=0.8\textwidth,
xlabel={kroki},
ylabel={Błąd},
/pgf/number format/.cd,
use comma,
1000 sep={}
]
\addplot[blue,semithick] file {wykresy/w_2_n_200_100.txt};
\end{axis}
\end{tikzpicture}
\caption{2 warstwy ukryte, (200, 100) neuronów}
\label{fig:w_1_n_200_100}
\end{figure}

\begin{figure}
\centering
\begin{tikzpicture}
\begin{axis}[
width=0.8\textwidth,
xlabel={kroki},
ylabel={Błąd},
/pgf/number format/.cd,
use comma,
1000 sep={}
]
\addplot[blue,semithick] file {wykresy/w_2_n_200_200.txt};
\end{axis}
\end{tikzpicture}
\caption{2 warstwy ukryte, (200, 200) neuronów}
\label{fig:w_1_n_200_200}
\end{figure}

\begin{figure}
\centering
\begin{tikzpicture}
\begin{axis}[
width=0.8\textwidth,
xlabel={kroki},
ylabel={Błąd},
/pgf/number format/.cd,
use comma,
1000 sep={}
]
\addplot[blue,semithick] file {wykresy/w_2_n_200_300.txt};
\end{axis}
\end{tikzpicture}
\caption{2 warstwy ukryte, (200, 300) neuronów}
\label{fig:w_1_n_200_300}
\end{figure}

\begin{figure}
\centering
\begin{tikzpicture}
\begin{axis}[
width=0.8\textwidth,
xlabel={kroki},
ylabel={Błąd},
/pgf/number format/.cd,
use comma,
1000 sep={}
]
\addplot[blue,semithick] file {wykresy/w_2_n_400_200.txt};
\end{axis}
\end{tikzpicture}
\caption{2 warstwy ukryte, (400, 200) neuronów}
\label{fig:w_2_n_400_200}
\end{figure}

\begin{figure}
\centering
\begin{tikzpicture}
\begin{axis}[
width=0.8\textwidth,
xlabel={kroki},
ylabel={Błąd},
/pgf/number format/.cd,
use comma,
1000 sep={}
]
\addplot[blue,semithick] file {wykresy/w_2_n_400_400.txt};
\end{axis}
\end{tikzpicture}
\caption{2 warstwy ukryte, (400, 400) neuronów}
\label{fig:w_2_n_400_400}
\end{figure}

\begin{figure}
\centering
\begin{tikzpicture}
\begin{axis}[
width=0.8\textwidth,
xlabel={kroki},
ylabel={Błąd},
/pgf/number format/.cd,
use comma,
1000 sep={}
]
\addplot[blue,semithick] file {wykresy/w_2_n_400_600.txt};
\end{axis}
\end{tikzpicture}
\caption{2 warstwy ukryte, (400, 600) neuronów}
\label{fig:w_2_n_400_600}
\end{figure}

\begin{figure}
\centering
\begin{tikzpicture}
\begin{axis}[
width=0.8\textwidth,
xlabel={kroki},
ylabel={Błąd},
/pgf/number format/.cd,
use comma,
1000 sep={}
]
\addplot[blue,semithick] file {wykresy/w_3_n_100_50_10.txt};
\end{axis}
\end{tikzpicture}
\caption{3 warstwy ukryte, (100, 50, 10) neuronów}
\label{fig:w_3_n_100_50_10}
\end{figure}

\begin{figure}
\centering
\begin{tikzpicture}
\begin{axis}[
width=0.8\textwidth,
xlabel={kroki},
ylabel={Błąd},
/pgf/number format/.cd,
use comma,
1000 sep={}
]
\addplot[blue,semithick] file {wykresy/w_3_n_100_100_50.txt};
\end{axis}
\end{tikzpicture}
\caption{3 warstwy ukryte, (100, 100, 50) neuronów}
\label{fig:w_3_n_100_100_50}
\end{figure}

\begin{figure}
\centering
\begin{tikzpicture}
\begin{axis}[
width=0.8\textwidth,
xlabel={kroki},
ylabel={Błąd},
/pgf/number format/.cd,
use comma,
1000 sep={}
]
\addplot[blue,semithick] file {wykresy/w_3_n_100_100_100.txt};
\end{axis}
\end{tikzpicture}
\caption{3 warstwy ukryte, (100, 100, 100) neuronów}
\label{fig:w_3_n_100_100_100}
\end{figure}

\begin{figure}
\centering
\begin{tikzpicture}
\begin{axis}[
width=0.8\textwidth,
xlabel={kroki},
ylabel={Błąd},
/pgf/number format/.cd,
use comma,
1000 sep={}
]
\addplot[blue,semithick] file {wykresy/w_3_n_200_100_100.txt};
\end{axis}
\end{tikzpicture}
\caption{3 warstwy ukryte, (200, 100, 100) neuronów}
\label{fig:w_3_n_200_100_100}
\end{figure}

\begin{figure}
\centering
\begin{tikzpicture}
\begin{axis}[
width=0.8\textwidth,
xlabel={kroki},
ylabel={Błąd},
/pgf/number format/.cd,
use comma,
1000 sep={}
]
\addplot[blue,semithick] file {wykresy/w_3_n_200_200_100.txt};
\end{axis}
\end{tikzpicture}
\caption{3 warstwy ukryte, (200, 200, 100) neuronów}
\label{fig:w_3_n_200_200_100}
\end{figure}

\begin{figure}
\centering
\begin{tikzpicture}
\begin{axis}[
width=0.8\textwidth,
xlabel={kroki},
ylabel={Błąd},
/pgf/number format/.cd,
use comma,
1000 sep={}
]
\addplot[blue,semithick] file {wykresy/w_3_n_200_200_200.txt};
\end{axis}
\end{tikzpicture}
\caption{3 warstwy ukryte, (200, 200, 200) neuronów}
\label{fig:w_3_n_200_200_200}
\end{figure}

\begin{figure}
\centering
\begin{tikzpicture}
\begin{axis}[
width=0.8\textwidth,
xlabel={kroki},
ylabel={Błąd},
/pgf/number format/.cd,
use comma,
1000 sep={}
]
\addplot[blue,semithick] file {wykresy/w_3_n_400_200_200.txt};
\end{axis}
\end{tikzpicture}
\caption{3 warstwy ukryte, (400, 200, 200) neuronów}
\label{fig:w_3_n_400_200_200}
\end{figure}
\section{Funkcja aktywacji}
Wyniki przedstawione w tabeli \ref{table:relu_sig} pozwalają jednoznacznie stwierdzić, że funkcja aktywacji ReLU daje znacznie lepsze wyniki niż sigmoidalna. Ponadto porównując wykresy \ref{fig:relu_test} i \ref{fig:sig_test} można zauważyć, że funkcja błędu ma znacznie mniejszą wariancję dla ReLU niż w przypadku funkcji sigmoidalnej. W kolejnych eksperymentach używana będzie funkcja aktywacji ReLU.
\begin{table}[h]
\centering
\begin{tabular}{|c|c|}
\hline
Funkcja aktywacji & Wyniki \\ hline
ReLU & \makecell{Zbiór uczący: 98,4\% \\ Zbiór testowy: 98\%} \\ \hline
Sigmoidalna & \makecell{Zbiór uczący: 90.6\% \\ Zbiór testowy: 94.2\%} \\ \hline
\end{tabular}
\caption{Porównanie działania funkcji aktywacji ReLU i sigmoidalnej}
\label{table:relu_sig}
\end{table}

\begin{figure}
\centering
\begin{tikzpicture}
\begin{axis}[
width=0.8\textwidth,
xlabel={kroki},
ylabel={Błąd},
/pgf/number format/.cd,
use comma,
1000 sep={}
]
\addplot[blue,semithick] file {wykresy/relu_test.txt};
\end{axis}
\end{tikzpicture}
\caption{Funkcja aktywacji ReLU}
\label{fig:relu_test}
\end{figure}

\begin{figure}
\centering
\begin{tikzpicture}
\begin{axis}[
width=0.8\textwidth,
xlabel={kroki},
ylabel={Błąd},
/pgf/number format/.cd,
use comma,
1000 sep={}
]
\addplot[blue,semithick] file {wykresy/sigmoid_test.txt};
\end{axis}
\end{tikzpicture}
\caption{Funkcja aktywacji sigmoidalna}
\label{fig:sig_test}
\end{figure}
\section{Algorytm uczenia}
Testy w poprzednich sekcjach były przeprowadzane z użyciem algorytmu SGD (Stochastic Gradient Descent). Tabela \ref{table:momentum} przedstawia wyniki dla algorytmu wykorzystującego pęd. Jak widać, dla niskich wartości parametru wyniki są praktycznie identyczne jak dla algorytmu SGD. Z kolei dla wyższych wartości parametru, wyniki okazują się być nieznacznie gorsze. Z uwagi na brak poprawy rezultatów mimo wprowadzenia dodatkowego parametru, kolejne testy będą przeprowadzane z wykorzystaniem algorytmu SGD.
\begin{table}
\centering
\begin{tabular}{|c|c|}
\hline
Parametr pędu & Wyniki \\ \hline
0,001 & \makecell{Zbiór uczący: 98,4\% \\ Zbiór testowy: 98\%} \\ \hline
0,005 & \makecell{Zbiór uczący: 98,4\% \\ Zbiór testowy: 98,1\%} \\ \hline
0,01 & \makecell{Zbiór uczący: 99,2\% \\ Zbiór testowy: 98\%} \\ \hline
0,05 & \makecell{Zbiór uczący: 97,7\% \\ Zbiór testowy: 98\%} \\ \hline
0,1 & \makecell{Zbiór uczący: 96,9\% \\ Zbiór testowy: 98\%} \\ \hline
0,2 & \makecell{Zbiór uczący: 96,7\% \\ Zbiór testowy: 98\%} \\ \hline
0,3 & \makecell{Zbiór uczący: 97,7\% \\ Zbiór testowy: 97,7\%} \\ \hline
0,3 & \makecell{Zbiór uczący: 97,7\% \\ Zbiór testowy: 97,5\%} \\ \hline
\end{tabular}
\caption{Wyniki dla algorytmu wykorzystującego pęd}
\label{table:momentum}
\end{table}

\begin{figure}
\centering
\begin{tikzpicture}
\begin{axis}[
width=0.8\textwidth,
xlabel={kroki},
ylabel={Błąd},
/pgf/number format/.cd,
use comma,
1000 sep={}
]
\addplot[blue,semithick] file {wykresy/mom_001.txt};
\end{axis}
\end{tikzpicture}
\caption{Parametr pędu - \num{0,001}}
\label{fig:mom_001}
\end{figure}

\begin{figure}
\centering
\begin{tikzpicture}
\begin{axis}[
width=0.8\textwidth,
xlabel={kroki},
ylabel={Błąd},
/pgf/number format/.cd,
use comma,
1000 sep={}
]
\addplot[blue,semithick] file {wykresy/mom_01.txt};
\end{axis}
\end{tikzpicture}
\caption{Parametr pędu - \num{0,01}}
\label{fig:mom_01}
\end{figure}

\begin{figure}
\centering
\begin{tikzpicture}
\begin{axis}[
width=0.8\textwidth,
xlabel={kroki},
ylabel={Błąd},
/pgf/number format/.cd,
use comma,
1000 sep={}
]
\addplot[blue,semithick] file {wykresy/mom_1.txt};
\end{axis}
\end{tikzpicture}
\caption{Parametr pędu - \num{0,1}}
\label{fig:mom_1}
\end{figure}

\begin{figure}
\centering
\begin{tikzpicture}
\begin{axis}[
width=0.8\textwidth,
xlabel={kroki},
ylabel={Błąd},
/pgf/number format/.cd,
use comma,
1000 sep={}
]
\addplot[blue,semithick] file {wykresy/mom_1.txt};
\end{axis}
\end{tikzpicture}
\caption{Parametr pędu - \num{0,2}}
\label{fig:mom_2}
\end{figure}

\begin{figure}
\centering
\begin{tikzpicture}
\begin{axis}[
width=0.8\textwidth,
xlabel={kroki},
ylabel={Błąd},
/pgf/number format/.cd,
use comma,
1000 sep={}
]
\addplot[blue,semithick] file {wykresy/mom_5.txt};
\end{axis}
\end{tikzpicture}
\caption{Parametr pędu - \num{0,4}}
\label{fig:mom_3}
\end{figure}
\section{Szybkość uczenia}
Tabela \ref{table:kroki} przedstawia wyniki dla różnych początkowych kroków uczenia. Krok jest w każdej iteracji zmniejszany według wzoru \ref{eq:exp_decay}, gdzie $i$ to aktualna iteracja algorytmu. 
\begin{equation}
\eta_i = \eta_{i-1} * 0,9^{\frac{i}{800}}
\label{eq:exp_decay}
\end{equation}

Testy przeprowadzanie w poprzednich testach miały początkową szybkość uczenia $\eta=\num{0,5}$. Tabela \ref{table:kroki} przedstawia wyniki dla innych wartości parametru. Jak można zauważyć, zmiana wartości kroku w przedziale od 0,2 do 0,6 nie zmieniła znacząco otrzymywanych wyników. Dla parametrów $\eta$ spoza tego przedziału następuje pogorszenie jakości klasyfikacji.
\begin{table}
\centering
\begin{tabular}{|c|c|}
\hline
Krok uczenia & Wyniki \\ \hline
0,01 & \makecell{Zbiór uczący: 90,6\% \\ Zbiór testowy: 94,2\%} \\ \hline
0,1 & \makecell{Zbiór uczący: 96,9\% \\ Zbiór testowy: 97,5\%} \\ \hline
0,2 & \makecell{Zbiór uczący: 97,7\% \\ Zbiór testowy: 97,9\%} \\ \hline
0,3 & \makecell{Zbiór uczący: 99,2\% \\ Zbiór testowy: 98\%} \\ \hline
0,4 & \makecell{Zbiór uczący: 98,4\% \\ Zbiór testowy: 98\%} \\ \hline
0,6 & \makecell{Zbiór uczący: 95,3\% \\ Zbiór testowy: 97,9\%} \\ \hline
0,7 & \makecell{Zbiór uczący: 96,9\% \\ Zbiór testowy: 97,4\%} \\ \hline

\end{tabular}
\caption{Wyniki dla różnych wartości parametru $\eta$}
\label{table:kroki}
\end{table}

\begin{figure}
\centering
\begin{tikzpicture}
\begin{axis}[
width=0.8\textwidth,
xlabel={kroki},
ylabel={Błąd},
/pgf/number format/.cd,
use comma,
1000 sep={}
]
\addplot[blue,semithick] file {wykresy/krok01.txt};
\end{axis}
\end{tikzpicture}
\caption{$\eta = \num{0,01}$}
\label{fig:krok_01}
\end{figure}

\begin{figure}
\centering
\begin{tikzpicture}
\begin{axis}[
width=0.8\textwidth,
xlabel={kroki},
ylabel={Błąd},
/pgf/number format/.cd,
use comma,
1000 sep={}
]
\addplot[blue,semithick] file {wykresy/krok1.txt};
\end{axis}
\end{tikzpicture}
\caption{$\eta = \num{0,1}$}
\label{fig:krok_1}
\end{figure}

\begin{figure}
\centering
\begin{tikzpicture}
\begin{axis}[
width=0.8\textwidth,
xlabel={kroki},
ylabel={Błąd},
/pgf/number format/.cd,
use comma,
1000 sep={}
]
\addplot[blue,semithick] file {wykresy/krok3.txt};
\end{axis}
\end{tikzpicture}
\caption{$\eta = \num{0,3}$}
\label{fig:krok_3}
\end{figure}

\begin{figure}
\centering
\begin{tikzpicture}
\begin{axis}[
width=0.8\textwidth,
xlabel={kroki},
ylabel={Błąd},
/pgf/number format/.cd,
use comma,
1000 sep={}
]
\addplot[blue,semithick] file {wykresy/krok4.txt};
\end{axis}
\end{tikzpicture}
\caption{$\eta = \num{0,4}$}
\label{fig:krok_4}
\end{figure}

\begin{figure}
\centering
\begin{tikzpicture}
\begin{axis}[
width=0.8\textwidth,
xlabel={kroki},
ylabel={Błąd},
/pgf/number format/.cd,
use comma,
1000 sep={}
]
\addplot[blue,semithick] file {wykresy/krok6.txt};
\end{axis}
\end{tikzpicture}
\caption{$\eta = \num{0,6}$}
\label{fig:krok_6}
\end{figure}

\begin{figure}
\centering
\begin{tikzpicture}
\begin{axis}[
width=0.8\textwidth,
xlabel={kroki},
ylabel={Błąd},
/pgf/number format/.cd,
use comma,
1000 sep={}
]
\addplot[blue,semithick] file {wykresy/krok75.txt};
\end{axis}
\end{tikzpicture}
\caption{$\eta = \num{0,7}$}
\label{fig:krok_7}
\end{figure}
\section{Batch size}
Poprzednie eksperymenty były przeprowadzane dla batch size (wielkości serii) równej 128. Tabela \ref{table:batchsize} przedstawia wyniki dla innych wartości tego parametru. Jak widać, mała wielkość serii powoduje spadek jakości klasyfikacji, co prawdopodobnie spowodowane jest tym, że sieć dopasowuje się do danych uczących.

Duży batch size polepsza działanie sieci neuronowej, ale sprawia, że nauka trwa kilkukrotnie dłużej. Wielkości 256 lub 512 stanowią dobry kompromis między szybkością nauki a dobrymi wynikami.

\begin{table}
\centering
\begin{tabular}{|c|c|}
\hline
Batch size & Wyniki \\ \hline
16 & \makecell{Zbiór uczący: 100\% \\ Zbiór testowy: 92,9\%} \\ \hline
32 & \makecell{Zbiór uczący: 100\% \\ Zbiór testowy: 96,8\%} \\ \hline
64 & \makecell{Zbiór uczący: 92,2\% \\ Zbiór testowy: 97,5\%} \\ \hline
256 & \makecell{Zbiór uczący: 99,6\% \\ Zbiór testowy: 98\%} \\ \hline
512 & \makecell{Zbiór uczący: 98,6\% \\ Zbiór testowy: 98,3\%} \\ \hline
1024 & \makecell{Zbiór uczący: 98,1\% \\ Zbiór testowy: 98,2\%} \\ \hline
\end{tabular}
\caption{Wyniki dla różnych wielkości serii}
\label{table:batchsize}
\end{table}

\begin{figure}
\centering
\begin{tikzpicture}
\begin{axis}[
width=0.8\textwidth,
xlabel={kroki},
ylabel={Błąd},
/pgf/number format/.cd,
use comma,
1000 sep={}
]
\addplot[blue,semithick] file {wykresy/batch16.txt};
\end{axis}
\end{tikzpicture}
\caption{Batch size = 16}
\label{fig:batch16}
\end{figure}

\begin{figure}
\centering
\begin{tikzpicture}
\begin{axis}[
width=0.8\textwidth,
xlabel={kroki},
ylabel={Błąd},
/pgf/number format/.cd,
use comma,
1000 sep={}
]
\addplot[blue,semithick] file {wykresy/batch32.txt};
\end{axis}
\end{tikzpicture}
\caption{Batch size = 32}
\label{fig:batch32}
\end{figure}

\begin{figure}
\centering
\begin{tikzpicture}
\begin{axis}[
width=0.8\textwidth,
xlabel={kroki},
ylabel={Błąd},
/pgf/number format/.cd,
use comma,
1000 sep={}
]
\addplot[blue,semithick] file {wykresy/batch64.txt};
\end{axis}
\end{tikzpicture}
\caption{Batch size = 64}
\label{fig:batch64}
\end{figure}

\begin{figure}
\centering
\begin{tikzpicture}
\begin{axis}[
width=0.8\textwidth,
xlabel={kroki},
ylabel={Błąd},
/pgf/number format/.cd,
use comma,
1000 sep={}
]
\addplot[blue,semithick] file {wykresy/batch256.txt};
\end{axis}
\end{tikzpicture}
\caption{Batch size = 256}
\label{fig:batch256}
\end{figure}

\begin{figure}
\centering
\begin{tikzpicture}
\begin{axis}[
width=0.8\textwidth,
xlabel={kroki},
ylabel={Błąd},
/pgf/number format/.cd,
use comma,
1000 sep={}
]
\addplot[blue,semithick] file {wykresy/batch512.txt};
\end{axis}
\end{tikzpicture}
\caption{Batch size = 512}
\label{fig:batch512}
\end{figure}

\begin{figure}
\centering
\begin{tikzpicture}
\begin{axis}[
width=0.8\textwidth,
xlabel={kroki},
ylabel={Błąd},
/pgf/number format/.cd,
use comma,
1000 sep={}
]
\addplot[blue,semithick] file {wykresy/batch1024.txt};
\end{axis}
\end{tikzpicture}
\caption{Batch size = 1024}
\label{fig:batch1024}
\end{figure}
\chapter{Wnioski końcowe}
Najlepszym wynikiem dla danych testowych, który udało nam się osiągnąć, było $\num{98,3}\%$ poprawnych odpowiedzi. Parametry sieci, która osiągnęła ten wynik, są następujące:
\begin{itemize}
\item struktura sieci - 2 warstwy ukryte po 400 neuronów
\item algorytm uczenia - Stochastic Gradient Descent
\item szybkość uczenia - \num{0,5}
\item funkcja aktywacji - ReLU
\item batch size - 512
\end{itemize}
\end{document}
