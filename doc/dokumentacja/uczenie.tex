\section{Szybkość uczenia}
Tabela \ref{table:kroki} przedstawia wyniki dla różnych początkowych kroków uczenia. Krok jest w każdej iteracji zmniejszany według wzoru \ref{eq:exp_decay}, gdzie $i$ to aktualna iteracja algorytmu. 
\begin{equation}
\eta_i = \eta_{i-1} * 0,9^{\frac{i}{800}}
\label{eq:exp_decay}
\end{equation}

Testy przeprowadzanie w poprzednich testach miały początkową szybkość uczenia $\eta=\num{0,5}$. Tabela \ref{table:kroki} przedstawia wyniki dla innych wartości parametru. Jak można zauważyć, zmiana wartości kroku w przedziale od 0,2 do 0,6 nie zmieniła znacząco otrzymywanych wyników. Dla parametrów $\eta$ spoza tego przedziału następuje pogorszenie jakości klasyfikacji.
\begin{table}
\centering
\begin{tabular}{|c|c|}
\hline
Krok uczenia & Wyniki \\ \hline
0,01 & \makecell{Zbiór uczący: 90,6\% \\ Zbiór testowy: 94,2\%} \\ \hline
0,1 & \makecell{Zbiór uczący: 96,9\% \\ Zbiór testowy: 97,5\%} \\ \hline
0,2 & \makecell{Zbiór uczący: 97,7\% \\ Zbiór testowy: 97,9\%} \\ \hline
0,3 & \makecell{Zbiór uczący: 99,2\% \\ Zbiór testowy: 98\%} \\ \hline
0,4 & \makecell{Zbiór uczący: 98,4\% \\ Zbiór testowy: 98\%} \\ \hline
0,6 & \makecell{Zbiór uczący: 95,3\% \\ Zbiór testowy: 97,9\%} \\ \hline
0,7 & \makecell{Zbiór uczący: 96,9\% \\ Zbiór testowy: 97,4\%} \\ \hline

\end{tabular}
\caption{Wyniki dla różnych wartości parametru $\eta$}
\label{table:kroki}
\end{table}

\begin{figure}
\centering
\begin{tikzpicture}
\begin{axis}[
width=0.8\textwidth,
xlabel={kroki},
ylabel={Błąd},
/pgf/number format/.cd,
use comma,
1000 sep={}
]
\addplot[blue,semithick] file {wykresy/krok01.txt};
\end{axis}
\end{tikzpicture}
\caption{$\eta = \num{0,01}$}
\label{fig:krok_01}
\end{figure}

\begin{figure}
\centering
\begin{tikzpicture}
\begin{axis}[
width=0.8\textwidth,
xlabel={kroki},
ylabel={Błąd},
/pgf/number format/.cd,
use comma,
1000 sep={}
]
\addplot[blue,semithick] file {wykresy/krok1.txt};
\end{axis}
\end{tikzpicture}
\caption{$\eta = \num{0,1}$}
\label{fig:krok_1}
\end{figure}

\begin{figure}
\centering
\begin{tikzpicture}
\begin{axis}[
width=0.8\textwidth,
xlabel={kroki},
ylabel={Błąd},
/pgf/number format/.cd,
use comma,
1000 sep={}
]
\addplot[blue,semithick] file {wykresy/krok3.txt};
\end{axis}
\end{tikzpicture}
\caption{$\eta = \num{0,3}$}
\label{fig:krok_3}
\end{figure}

\begin{figure}
\centering
\begin{tikzpicture}
\begin{axis}[
width=0.8\textwidth,
xlabel={kroki},
ylabel={Błąd},
/pgf/number format/.cd,
use comma,
1000 sep={}
]
\addplot[blue,semithick] file {wykresy/krok4.txt};
\end{axis}
\end{tikzpicture}
\caption{$\eta = \num{0,4}$}
\label{fig:krok_4}
\end{figure}

\begin{figure}
\centering
\begin{tikzpicture}
\begin{axis}[
width=0.8\textwidth,
xlabel={kroki},
ylabel={Błąd},
/pgf/number format/.cd,
use comma,
1000 sep={}
]
\addplot[blue,semithick] file {wykresy/krok6.txt};
\end{axis}
\end{tikzpicture}
\caption{$\eta = \num{0,6}$}
\label{fig:krok_6}
\end{figure}

\begin{figure}
\centering
\begin{tikzpicture}
\begin{axis}[
width=0.8\textwidth,
xlabel={kroki},
ylabel={Błąd},
/pgf/number format/.cd,
use comma,
1000 sep={}
]
\addplot[blue,semithick] file {wykresy/krok75.txt};
\end{axis}
\end{tikzpicture}
\caption{$\eta = \num{0,7}$}
\label{fig:krok_7}
\end{figure}