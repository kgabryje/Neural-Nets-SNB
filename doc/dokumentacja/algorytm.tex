\section{Algorytm uczenia}
Testy w poprzednich sekcjach były przeprowadzane z użyciem algorytmu SGD (Stochastic Gradient Descent). Tabela \ref{table:momentum} przedstawia wyniki dla algorytmu wykorzystującego pęd. Jak widać, dla niskich wartości parametru wyniki są praktycznie identyczne jak dla algorytmu SGD. Z kolei dla wyższych wartości parametru, wyniki okazują się być nieznacznie gorsze. Z uwagi na brak poprawy rezultatów mimo wprowadzenia dodatkowego parametru, kolejne testy będą przeprowadzane z wykorzystaniem algorytmu SGD.
\begin{table}
\centering
\begin{tabular}{|c|c|}
\hline
Parametr pędu & Wyniki \\ \hline
0,001 & \makecell{Zbiór uczący: 98,4\% \\ Zbiór testowy: 98\%} \\ \hline
0,005 & \makecell{Zbiór uczący: 98,4\% \\ Zbiór testowy: 98,1\%} \\ \hline
0,01 & \makecell{Zbiór uczący: 99,2\% \\ Zbiór testowy: 98\%} \\ \hline
0,05 & \makecell{Zbiór uczący: 97,7\% \\ Zbiór testowy: 98\%} \\ \hline
0,1 & \makecell{Zbiór uczący: 96,9\% \\ Zbiór testowy: 98\%} \\ \hline
0,2 & \makecell{Zbiór uczący: 96,7\% \\ Zbiór testowy: 98\%} \\ \hline
0,3 & \makecell{Zbiór uczący: 97,7\% \\ Zbiór testowy: 97,7\%} \\ \hline
0,3 & \makecell{Zbiór uczący: 97,7\% \\ Zbiór testowy: 97,5\%} \\ \hline
\end{tabular}
\caption{Wyniki dla algorytmu wykorzystującego pęd}
\label{table:momentum}
\end{table}

\begin{figure}
\centering
\begin{tikzpicture}
\begin{axis}[
width=0.8\textwidth,
xlabel={kroki},
ylabel={Błąd},
/pgf/number format/.cd,
use comma,
1000 sep={}
]
\addplot[blue,semithick] file {wykresy/mom_001.txt};
\end{axis}
\end{tikzpicture}
\caption{Parametr pędu - \num{0,001}}
\label{fig:mom_001}
\end{figure}

\begin{figure}
\centering
\begin{tikzpicture}
\begin{axis}[
width=0.8\textwidth,
xlabel={kroki},
ylabel={Błąd},
/pgf/number format/.cd,
use comma,
1000 sep={}
]
\addplot[blue,semithick] file {wykresy/mom_01.txt};
\end{axis}
\end{tikzpicture}
\caption{Parametr pędu - \num{0,01}}
\label{fig:mom_01}
\end{figure}

\begin{figure}
\centering
\begin{tikzpicture}
\begin{axis}[
width=0.8\textwidth,
xlabel={kroki},
ylabel={Błąd},
/pgf/number format/.cd,
use comma,
1000 sep={}
]
\addplot[blue,semithick] file {wykresy/mom_1.txt};
\end{axis}
\end{tikzpicture}
\caption{Parametr pędu - \num{0,1}}
\label{fig:mom_1}
\end{figure}

\begin{figure}
\centering
\begin{tikzpicture}
\begin{axis}[
width=0.8\textwidth,
xlabel={kroki},
ylabel={Błąd},
/pgf/number format/.cd,
use comma,
1000 sep={}
]
\addplot[blue,semithick] file {wykresy/mom_1.txt};
\end{axis}
\end{tikzpicture}
\caption{Parametr pędu - \num{0,2}}
\label{fig:mom_2}
\end{figure}

\begin{figure}
\centering
\begin{tikzpicture}
\begin{axis}[
width=0.8\textwidth,
xlabel={kroki},
ylabel={Błąd},
/pgf/number format/.cd,
use comma,
1000 sep={}
]
\addplot[blue,semithick] file {wykresy/mom_5.txt};
\end{axis}
\end{tikzpicture}
\caption{Parametr pędu - \num{0,4}}
\label{fig:mom_3}
\end{figure}