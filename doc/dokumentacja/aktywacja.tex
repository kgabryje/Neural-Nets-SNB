\section{Funkcja aktywacji}
Wyniki przedstawione w tabeli \ref{table:relu_sig} pozwalają jednoznacznie stwierdzić, że funkcja aktywacji ReLU daje znacznie lepsze wyniki niż sigmoidalna. Ponadto porównując wykresy \ref{fig:relu_test} i \ref{fig:sig_test} można zauważyć, że funkcja błędu ma znacznie mniejszą wariancję dla ReLU niż w przypadku funkcji sigmoidalnej. W kolejnych eksperymentach używana będzie funkcja aktywacji ReLU.
\begin{table}[h]
\centering
\begin{tabular}{|c|c|}
\hline
Funkcja aktywacji & Wyniki \\ \hline
ReLU & \makecell{Zbiór uczący: 98,4\% \\ Zbiór testowy: 98\%} \\ \hline
Sigmoidalna & \makecell{Zbiór uczący: 90.6\% \\ Zbiór testowy: 94.2\%} \\ \hline
\end{tabular}
\caption{Porównanie działania funkcji aktywacji ReLU i sigmoidalnej}
\label{table:relu_sig}
\end{table}

\begin{figure}
\centering
\begin{tikzpicture}
\begin{axis}[
width=0.8\textwidth,
xlabel={kroki},
ylabel={Błąd},
/pgf/number format/.cd,
use comma,
1000 sep={}
]
\addplot[blue,semithick] file {wykresy/relu_test.txt};
\end{axis}
\end{tikzpicture}
\caption{Funkcja aktywacji ReLU}
\label{fig:relu_test}
\end{figure}

\begin{figure}
\centering
\begin{tikzpicture}
\begin{axis}[
width=0.8\textwidth,
xlabel={kroki},
ylabel={Błąd},
/pgf/number format/.cd,
use comma,
1000 sep={}
]
\addplot[blue,semithick] file {wykresy/sigmoid_test.txt};
\end{axis}
\end{tikzpicture}
\caption{Funkcja aktywacji sigmoidalna}
\label{fig:sig_test}
\end{figure}