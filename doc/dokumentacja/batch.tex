\section{Batch size}
Poprzednie eksperymenty były przeprowadzane dla batch size (wielkości serii) równej 128. Tabela \ref{table:batchsize} przedstawia wyniki dla innych wartości tego parametru. Jak widać, mała wielkość serii powoduje spadek jakości klasyfikacji, co prawdopodobnie spowodowane jest tym, że sieć dopasowuje się do danych uczących.

Duży batch size polepsza działanie sieci neuronowej, ale sprawia, że nauka trwa kilkukrotnie dłużej. Wielkości 256 lub 512 stanowią dobry kompromis między szybkością nauki a dobrymi wynikami.

\begin{table}
\centering
\begin{tabular}{|c|c|}
\hline
Batch size & Wyniki \\ \hline
16 & \makecell{Zbiór uczący: 100\% \\ Zbiór testowy: 92,9\%} \\ \hline
32 & \makecell{Zbiór uczący: 100\% \\ Zbiór testowy: 96,8\%} \\ \hline
64 & \makecell{Zbiór uczący: 92,2\% \\ Zbiór testowy: 97,5\%} \\ \hline
256 & \makecell{Zbiór uczący: 99,6\% \\ Zbiór testowy: 98\%} \\ \hline
512 & \makecell{Zbiór uczący: 98,6\% \\ Zbiór testowy: 98,3\%} \\ \hline
1024 & \makecell{Zbiór uczący: 98,1\% \\ Zbiór testowy: 98,2\%} \\ \hline
\end{tabular}
\caption{Wyniki dla różnych wielkości serii}
\label{table:batchsize}
\end{table}

\begin{figure}
\centering
\begin{tikzpicture}
\begin{axis}[
width=0.8\textwidth,
xlabel={kroki},
ylabel={Błąd},
/pgf/number format/.cd,
use comma,
1000 sep={}
]
\addplot[blue,semithick] file {wykresy/batch16.txt};
\end{axis}
\end{tikzpicture}
\caption{Batch size = 16}
\label{fig:batch16}
\end{figure}

\begin{figure}
\centering
\begin{tikzpicture}
\begin{axis}[
width=0.8\textwidth,
xlabel={kroki},
ylabel={Błąd},
/pgf/number format/.cd,
use comma,
1000 sep={}
]
\addplot[blue,semithick] file {wykresy/batch32.txt};
\end{axis}
\end{tikzpicture}
\caption{Batch size = 32}
\label{fig:batch32}
\end{figure}

\begin{figure}
\centering
\begin{tikzpicture}
\begin{axis}[
width=0.8\textwidth,
xlabel={kroki},
ylabel={Błąd},
/pgf/number format/.cd,
use comma,
1000 sep={}
]
\addplot[blue,semithick] file {wykresy/batch64.txt};
\end{axis}
\end{tikzpicture}
\caption{Batch size = 64}
\label{fig:batch64}
\end{figure}

\begin{figure}
\centering
\begin{tikzpicture}
\begin{axis}[
width=0.8\textwidth,
xlabel={kroki},
ylabel={Błąd},
/pgf/number format/.cd,
use comma,
1000 sep={}
]
\addplot[blue,semithick] file {wykresy/batch256.txt};
\end{axis}
\end{tikzpicture}
\caption{Batch size = 256}
\label{fig:batch256}
\end{figure}

\begin{figure}
\centering
\begin{tikzpicture}
\begin{axis}[
width=0.8\textwidth,
xlabel={kroki},
ylabel={Błąd},
/pgf/number format/.cd,
use comma,
1000 sep={}
]
\addplot[blue,semithick] file {wykresy/batch512.txt};
\end{axis}
\end{tikzpicture}
\caption{Batch size = 512}
\label{fig:batch512}
\end{figure}

\begin{figure}
\centering
\begin{tikzpicture}
\begin{axis}[
width=0.8\textwidth,
xlabel={kroki},
ylabel={Błąd},
/pgf/number format/.cd,
use comma,
1000 sep={}
]
\addplot[blue,semithick] file {wykresy/batch1024.txt};
\end{axis}
\end{tikzpicture}
\caption{Batch size = 1024}
\label{fig:batch1024}
\end{figure}